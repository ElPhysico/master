%************************************************
\chapter{Search and Navigation in the Presence of Hints}\label{ch:hints}
%************************************************
In all previous considerations of search problems we assumed the walker to be completely blind, \ie the search is purely random. However, as we have mentioned before in \autoref{sec:class-search-strategies}, sometimes there are hints in the environment that contain valuable information of the whereabouts of the target. If the searcher is capable of perceiving and analyzing these hints, it can use the additional information in order to adapt its strategy. Some examples include \eg humans reading tracks or dogs sensing odors. However, the perceiving of hints during search is not restricted to more advanced life forms like humans and animals, but can also be observed in simpler organisms like bacteria and cells. In this chapter we will introduce hint-related navigation and search strategies of microorganisms, and we will present related current research.

\section{Directional navigation (Taxis)}
Different organisms are able to detect or sense different types of hints or stimuli in their environment. In the case of \textit{taxis}, the organism responds to a stimulus with directional movement, which means that it moves towards or away from the stimuli source\graffito{Taxis describes guided navigation as a response to a stimulus.}. If the stimulus induces movement towards the source, one speaks of positive taxis, whereas movement away from the source is denoted by negative taxis. In particular, this means that an organism featuring taxis at least roughly knows in which direction the source of the stimuli is located.

Taxes are generally categorized based on the type of stimulus which induces the response of the organism. Up to today, many taxes have been identified, including:
\begin{description}
 \item[aerotaxis] response to oxygen concentration gradients,
 \item[barotaxis] pressure gradients,
 \item[chemotaxis] chemical concentration gradients,
 \item[durotaxis] stiffness gradients,
 \item[electrotaxis] electric field,
 \item[gravitaxis] gravity field,
 \item[magnetotaxis] magnetic field,
 \item[phototaxis] light intensity and direction,
 \item[rheotaxis] current in a fluid,
 \item[thermotaxis] temperature gradient.
\end{description}
From this list, especially chemotaxis plays an important role in the movement and navigation of different cells and bacteria.

\subsection{Chemotaxis}

\section{Non-directional navigation (Kinesis)}





%*****************************************
%*****************************************
%*****************************************
%*****************************************
%*****************************************
